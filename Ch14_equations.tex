% Never doing this again...probably

\documentclass{report}  % For chapter number
\usepackage{amsmath, chngcntr}

\counterwithin{equation}{section}  % Reset equation counter every section
\counterwithout{equation}{chapter} 
\setlength\parindent{0pt}  % No indent

\begin{document}
\setcounter{chapter}{14}
\section{Functions of Several Variables}
\subsection*{New Definitions}
\paragraph{Level Curves} 
 $f(x,y)=k$ where $k$ is a constant.
 
\paragraph{Level Surfaces}
$f(x,y,z)=k$

\section{Limits and Continuity}
\subsection*{Definition of Limits} 
Let $f$ be a function of two variables whose domain $D$ includes points arbitrarily close to $(a, b)$. 

\begin{equation}
	\lim_{(x,y) \to (a,b)} f(x, y) = L
\end{equation}
if for every number $\varepsilon > 0$ there is a corresponding number $\delta > 0$ such that 

\begin{equation}
	\text{if} \quad (x,y) \in D  \ \text{and} \ 0 < \sqrt{(x-a)^2 + (y-b)^2} < \delta \quad 
	\text{then} \quad |f(x,y) - L| < \varepsilon
\end{equation}

 
\subsection*{When the Limit Does Not Exist} 
If we can find two different paths of approach along which the function $f(x, y)$ has different limits, then it follows that $\lim_{(x, y) \to (a, b)} f(x, y)$ does not exist. 

\paragraph{Limit Laws} Approaching along axes and a constant
\begin{equation}
	\lim_{(x,y) \to (a,b)} x = a \qquad
	\lim_{(x,y) \to (a,b)} y = b \qquad
	\lim_{(x,y) \to (a,b)} c = c
\end{equation}
The \textbf{Squeeze Theorem} also holds.

\subsection*{Continuity} 
The direct substitution property: A function $f$ is called \\
\textbf{continuous at} $(a, b)$ if 
\begin{equation}
	\lim_{(x,y) \to (a,b)} f(x, y) = f(a, b)
\end{equation}
We say $f$ is \textbf{continuous on} $D$ if $f$ is continuous at every point $(a, b)$ in $D$. 

\newpage

\section{Partial Derivatives}
\subsection*{Partial Derivatives}
\begin{equation}
	f_x (a, b) = g'(a) \quad \text{where} \quad g{x} = f(x, b)
\end{equation}	

\subsection*{Definition of Derivatives}
The partial derivatives are defined by
\begin{equation}
\begin{aligned}
	f_x(x,y) = \lim_{h \to 0} \frac{f(x+h,y)-f(x,y)}{h} \\
	f_y(x,y) = \lim_{h \to 0} \frac{f(x,y+h)-f(x,y)}{h}
\end{aligned}
\end{equation}

\subsection*{Notation}
If $z = f(x, y)$
\begin{equation*}  % Here we go
\begin{aligned}
	f_x(x,y) = f_x = \frac{\partial f}{\partial x} = 
	\frac{\partial }{\partial x} f(x, y) = \frac{\partial z}{\partial x} = 
	f_1 = D_1 f = D_x f	\\ \\
	f_y(x,y) = f_y = \frac{\partial f}{\partial y} = 
	\frac{\partial }{\partial y} f(x, y) = \frac{\partial z}{\partial y} = 
	f_2 = D_2 f = D_y f	
\end{aligned}
\end{equation*}

\subsection*{Finding Partial Derivatives}
To find $f_x$, treat $y$ as a constant and differentiate $f(x, y)$ with respect to $x$. \\
To find $f_y$, treat $x$ as a constant and differentiate $f(x, y)$ with respect to $y$.

\subsection*{Interpretations}
The partial derivatives $f_x(a, b)$ and $f_y(a, b)$ can be interpreted geometrically as the slopes of the tangent lines at $P(a, b, c)$ to the traces $C_1$ and $C_2$ of $S$ in the planes $y=b$ and $x=a$. 

Partial derivatives can also be interpreted as \textit{rates of change}. If $z = f(x,y)$, then $\partial z / \partial x$ represents the rate of change of $z$ with respect to $x$ when $y$ is fixed. 
Similarly, $\partial z / \partial y$ represents the rate of change of $z$ with respect to $y$ when $x$ is fixed.

\subsection*{More Than Two Variables}
Treated similarly to two variable functions. If $w = f(x, y, z)$, then $f_x = \partial w / \partial x$ with respect to $x$ when $y$ and $z$ are held constant. Similar notation follows. 

\subsection*{Higher Derivatives}
\paragraph{Second Partial Derivatives} The partial derivatives of the partial derivatives, $(f_x)_x, (f_x)_y, (f_y)_x, (f_y)_y$. The notation is similar to the notation of higher derivatives of functions of one variable.

\paragraph{Clairaut's Theorem} Suppose $f$ is defined on a disk $D$ that contains the point $(a, b)$. If the functions $f_{xy}$ and $f_{yx}$ are both continuous on $D$, then 
\begin{equation}
	f_{xy}(a, b) = f_{yx}(a, b)
\end{equation}

\newpage

\section{Tangent Planes and Linear Approximations}
\subsection*{Tangent Planes}
Suppose a surface $S$ had equation $z=f(x,y)$ with continuous first derivatives.
Let $C_1$ and $C_2$ be the curves obtained by intersecting the vertical planes $y=y_0$ and $x=x_0$ with the surface $S$. 
Let $T_1$ and $T_2$ be the tangent lines to the curves $C_1$ and $C_2$ at $P$. 

\paragraph{Tangent Plane} The plane to $S$ at $P$ that contains tangent lines $T_1$ and $T_2$. \\ \\
At $P(x_0,y_0,z_0)$
\begin{equation}
	z - z_0 = f_x(x_0,y_0)(x-x_0) + f_y(x_0,y_0)(y-y_0)
\end{equation}

\subsection*{Linear Approximations}
At point $(a, b, f(a, b))$, the \textbf{linearlization} is
\begin{equation}
	z = L(x, y) = f(a, b) + f_x(a,b)(x-a) + f_y(a,b)(y-b)
\end{equation}
The approximation of $f(x, y)$ is called the \textbf{linear approximation}.

\subsection*{Differentiable Functions}
\paragraph{Theorem for differentiability} If the partial derivatives $f_x$ and $f_y$ exist near $(a, b)$ and are continuous at $(a, b)$, then $f$ is differentiable at $(a, b)$. \\ 

Even if \textit{directional derivatives} exist at a point in every direction, the function still may not be differentiable at that point. 
But if a function is differentiable at a point, then \textit{all} directional derivatives will exist at that point. 

\subsection*{Three or More Variables}
Defined similarly.

\newpage

\section{The Chain Rule}
\subsection*{Chain Rule for Functions of Two Variables}
\paragraph{Chain Rule Case 1} Suppose $z = f(x, y), \ x = g(t), \ y = h(t)$, and all are differentiable. Then

\begin{equation}
	\frac{dz}{dt} = \frac{\partial z}{\partial x} \frac{dx}{dt} + \frac{\partial z}{\partial y} \frac{dy}{dt}
\end{equation}

\paragraph{Chain Rule Case 2} Suppose $z = f(x, y), \ x = g(s, t), \ y = h(s, t)$, and all are differentiable. Then

\begin{equation}
	\frac{dz}{ds} = \frac{\partial z}{\partial x} \frac{\partial x}{ \partial s} + \frac{\partial z}{\partial y} \frac{\partial y}{\partial s}
	\qquad \quad
	\frac{dz}{dt} = \frac{\partial z}{\partial x} \frac{\partial x}{ \partial t} + \frac{\partial z}{\partial y} \frac{\partial y}{\partial t}
\end{equation}
\\

A tree diagram can help remember the Chain Rule (supposedly). 

\subsection*{Implicit Differentiation}
Writing $F(x, y) = 0$ which defines $y$ implicitly in terms of $x$, that is $y = f(x)$ so $F(x, f(x)) = 0$, then 

\begin{equation}
	% Extra slashes in numerator to make center fraction line longer
	\frac{dy}{dx} = - \displaystyle \frac{\displaystyle  \ \frac{\partial F}{\partial x } \ }{\displaystyle \frac{\partial F}{\partial y}}
	= - \frac{F_x}{F_y}
\end{equation} \\

For $z$ given implicitly as $z = f(x, y)$ by $F(x, y, z) = 0$, that is $F(x, y, f(x, y)) = 0$, we get similar formulas: 

\begin{equation}
	\frac{\partial z}{\partial x} = - \frac{F_x}{F_z}
	\qquad \quad
	\frac{\partial z}{\partial y} = - \frac{F_y}{F_z}
\end{equation}

Note in both cases the fraction appears inverted, but that is a consequence of solving for the implicit derivative(s).

\newpage

\section{Directional Derivatives and the Gradient Vector}
\subsection*{Directional Derivatives}

For differentiable $f$, $f$ has a directional derivative in the direction of the unit vector $\vec{u} = \langle a, b \rangle$ and 
\begin{equation}
	D_{\vec{u}} f(x, y) = f_x (x, y)a + f_y (x, y)b
\end{equation}

If $\vec{u}$ makes angle $\theta$ with the positive x-axis, we can write $\vec{u} = \langle \cos \theta , \sin \theta \rangle$.

\paragraph{The Gradient Vector}
Since $D_u f(x, y) = \langle f_x (x, y), f_y (x, y) \rangle \cdot \langle a, b \rangle$,
we give the first vector a special name, \textbf{grad} $f$ or $\nabla f$ ("del" or "nabla"). 

\begin{equation}
	\nabla f(x, y) = \langle f_x (x, y), f_y (x, y) \rangle
\end{equation}

\begin{equation}
	D_u f(x, y) = \nabla f(x, y) \cdot \vec{u}
\end{equation}

\subsection*{Functions of Three Variables}
Defined similarly. 

\begin{equation}
	\nabla f= \langle f_x , f_y , f_z \rangle
\end{equation}

\begin{equation}
	D_u f(x, y, z) = \nabla f(x, y, z) \cdot \vec{u}
\end{equation}

\subsection*{Maximizing the Directional Derivatives}
Suppose $f$ is differentiable. 
\begin{equation}
	\max(D_u f(\vec{x})) = |\nabla f(\vec{x})|
\end{equation}
This occurs when $\vec{u}$ has the same direction as $\nabla f(\vec{x})$, or in my own notation

\begin{equation}
	\vec{u} = k \nabla f(\vec{x})
\end{equation}
Normalize $\nabla f(\vec{x})$ to get $\vec{u}$. 

\subsection*{Tangent Planes to Level Surfaces}
If $\nabla F(x_0, y_0, z_0) \ne 0$, define the \textbf{tangent plane to the level surface} $F(x, y, z) = k$ at $P(x_0, y_0, z_0)$ as the plane that passes through $P$ and has normal vector $\nabla F(x_0, y_0, z_0)$. 

Using the standard equation of a plane, we can write 
\begin{equation}
	F_x (x_0, y_0, z_0) (x-x_0) + 
	F_y (x_0, y_0, z_0) (y-y_0) + 
	F_z (x_0, y_0, z_0) (z-z_0) = 0
\end{equation} \\

The \textbf{normal line} is perpendicular to the tangent plane and given by 
$\nabla F(x_0, y_0, z_0)$. It has symmetric equations
\begin{equation}
	\frac{x-x_0}{F_x (x_0, y_0, z_0)} = 
	\frac{y-y_0}{F_y (x_0, y_0, z_0)} = 
	\frac{z-z_0}{F_z (x_0, y_0, z_0)}
\end{equation}
\end{document}