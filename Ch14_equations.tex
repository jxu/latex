% Basic formatting

\documentclass{report}  % For chapter number
\usepackage{amsmath, chngcntr}

\counterwithin{equation}{section}  % Reset equation counter every section
\counterwithout{equation}{chapter} 
\setlength\parindent{0pt}  % No indent

\begin{document}
\setcounter{chapter}{14}
\section{Functions of Several Variables}
\subsection*{New Definitions}
\paragraph{Level Curves} 
 $f(x,y)=k$ where $k$ is a constant.
 
\paragraph{Level Surfaces}
$f(x,y,z)=k$

\section{Limits and Continuity}
\subsection*{Definition of Limits} 
Let $f$ be a function of two variables whose domain $D$ includes points arbitrarily close to $(a, b)$. 

\begin{equation}
	\lim_{(x,y) \to (a,b)} f(x, y) = L
\end{equation}
if for every number $\varepsilon > 0$ there is a corresponding number $\delta > 0$ such that 

\begin{equation}
	\text{if} \quad (x,y) \in D  \ \text{and} \ 0 < \sqrt{(x-a)^2 + (y-b)^2} < \delta \quad 
	\text{then} \quad |f(x,y) - L| < \varepsilon
\end{equation}

 
\subsection*{When the Limit Does Not Exist} 
If we can find two different paths of approach along which the function $f(x, y)$ has different limits, then it follows that $\lim_{(x, y) \to (a, b)} f(x, y)$ does not exist. 

\paragraph{Limit Laws} Approaching along axes and a constant
\begin{equation}
	\lim_{(x,y) \to (a,b)} x = a \qquad
	\lim_{(x,y) \to (a,b)} y = b \qquad
	\lim_{(x,y) \to (a,b)} c = c
\end{equation}
The \textbf{Squeeze Theorem} also holds.

\subsection*{Continuity} 
The direct substitution property: A function $f$ is called \\
\textbf{continuous at} $(a, b)$ if 
\begin{equation}
	\lim_{(x,y) \to (a,b)} f(x, y) = f(a, b)
\end{equation}
We say $f$ is \textbf{continuous on} $D$ if $f$ is continuous at every point $(a, b)$ in $D$. 

\newpage

\section{Partial Derivatives}
\subsection*{Partial Derivatives}
\begin{equation}
	f_x (a, b) = g'(a) \quad \text{where} \quad g{x} = f(x, b)
\end{equation}	

\subsection*{Definition of Derivatives}
The partial derivatives are defined by
\begin{equation}
\begin{aligned}
	f_x(x,y) = \lim_{h \to 0} \frac{f(x+h,y)-f(x,y)}{h} \\
	f_y(x,y) = \lim_{h \to 0} \frac{f(x,y+h)-f(x,y)}{h}
\end{aligned}
\end{equation}

\subsection*{Notation}
If $z = f(x, y)$
\begin{equation*}  % Here we go
\begin{aligned}
	f_x(x,y) = f_x = \frac{\partial f}{\partial x} = 
	\frac{\partial }{\partial x} f(x, y) = \frac{\partial z}{\partial x} = 
	f_1 = D_1 f = D_x f	\\ \\
	f_y(x,y) = f_y = \frac{\partial f}{\partial y} = 
	\frac{\partial }{\partial y} f(x, y) = \frac{\partial z}{\partial y} = 
	f_2 = D_2 f = D_y f	
\end{aligned}
\end{equation*}

\subsection*{Finding Partial Derivatives}
To find $f_x$, treat $y$ as a constant and differentiate $f(x, y)$ with respect to $x$. \\
To find $f_y$, treat $x$ as a constant and differentiate $f(x, y)$ with respect to $y$.

\subsection*{Interpretations}
The partial derivatives $f_x(a, b)$ and $f_y(a, b)$ can be interpreted geometrically as the slopes of the tangent lines at $P(a, b, c)$ to the traces $C_1$ and $C_2$ of $S$ in the planes $y=b$ and $x=a$. 

Partial derivatives can also be interpreted as \textit{rates of change}. If $z = f(x,y)$, then $\partial z / \partial x$ represents the rate of change of $z$ with respect to $x$ when $y$ is fixed. 
Similarly, $\partial z / \partial y$ represents the rate of change of $z$ with respect to $y$ when $x$ is fixed.

\subsection*{More Than Two Variables}
Treated similarly to two variable functions. If $w = f(x, y, z)$, then $f_x = \partial w / \partial x$ with respect to $x$ when $y$ and $z$ are held constant. Similar notation follows. 

\subsection*{Higher Derivatives}
\paragraph{Second Partial Derivatives} The partial derivatives of the partial derivatives, $(f_x)_x, (f_x)_y, (f_y)_x, (f_y)_y$. The notation is similar to the notation of higher derivatives of functions of one variable.

\paragraph{Clairaut's Theorem} Suppose $f$ is defined on a disk $D$ that contains the point $(a, b)$. If the functions $f_{xy}$ and $f_{yx}$ are both continuous on $D$, then 
\begin{equation}
	f_{xy}(a, b) = f_{yx}(a, b)
\end{equation}

\newpage

\section{Tangent Planes and Linear Approximations}
\subsection*{Tangent Planes}
Suppose a surface $S$ had equation $z=f(x,y)$ with continuous first derivatives.
Let $C_1$ and $C_2$ be the curves obtained by intersecting the vertical planes $y=y_0$ and $x=x_0$ with the surface $S$. 
Let $T_1$ and $T_2$ be the tangent lines to the curves $C_1$ and $C_2$ at $P$. 

\paragraph{Tangent Plane} The plane to $S$ at $P$ that contains tangent lines $T_1$ and $T_2$. \\ \\
At $P(x_0,y_0,z_0)$
\begin{equation}
	z - z_0 = f_x(x_0,y_0)(x-x_0) + f_y(x_0,y_0)(y-y_0)
\end{equation}

\subsection*{Linear Approximations}
At point $(a, b, f(a, b))$, the \textbf{linearlization} is
\begin{equation}
	z = L(x, y) = f(a, b) + f_x(a,b)(x-a) + f_y(a,b)(y-b)
\end{equation}
The approximation of $f(x, y)$ is called the \textbf{linear approximation}.

\subsection*{Differentiable Functions}
\paragraph{Theorem for differentiability} If the partial derivatives $f_x$ and $f_y$ exist near $(a, b)$ and are continuous at $(a, b)$, then $f$ is differentiable at $(a, b)$. \\ 

Even if \textit{directional derivatives} exist at a point in every direction, the function still may not be differentiable at that point. 
But if a function is differentiable at a point, then \textit{all} directional derivatives will exist at that point. 

\subsection*{Three or More Variables}
Defined similarly.

\end{document}