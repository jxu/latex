% Basic formatting

\documentclass{article}
\usepackage{amsmath, amsthm}  % amsthm not used

\begin{document}
\section*{14.1 Functions of Several Variables}
\paragraph{Level Curves} 
 $f(x,y)=k$ where $k$ is a constant.
 
\paragraph{Level Surfaces}
$f(x,y,z)=k$

\section*{14.2 Limits and Continuity}
\subsection*{Definition of limits} 
Let $f$ be a function of two variables whose domain $D$ includes points arbitrarily close to $(a, b)$. 

\begin{equation}
	\lim_{(x,y) \to (a,b)} f(x, y) = L
\end{equation}
if for every number $\varepsilon > 0$ there is a corresponding number $\delta > 0$ such that 

\begin{equation}
	\text{if} \quad (x,y) \in D  \ \text{and} \ 0 < \sqrt{(x-a)^2 + (y-b)^2} < \delta \quad 
	\text{then} \quad |f(x,y) - L| < \varepsilon
\end{equation}

 
\subsection*{When the Limit Does Not Exist} 
If we can find two different paths of approach along which the function $f(x, y)$ has different limits, then it follows that $\lim_{(x, y) \to (a, b)} f(x, y)$ does not exist. 

\paragraph{Limit Laws} Approaching along axes and a constant
\begin{equation}
	\lim_{(x,y) \to (a,b)} x = a \qquad
	\lim_{(x,y) \to (a,b)} y = b \qquad
	\lim_{(x,y) \to (a,b)} c = c
\end{equation}
The \textbf{Squeeze Theorem} also holds.

\subsection*{Continuity} 
The direct substitution property: A function $f$ is called \\
\textbf{continuous at} $(a, b)$ if 
\begin{equation}
	\lim_{(x,y) \to (a,b)} f(x, y) = f(a, b)
\end{equation}
We say $f$ is \textbf{continuous on} $D$ if $f$ is continuous at every point $(a, b)$ in $D$. 

\newpage

\section*{14.3 Partial Derivatives}

	

\end{document}