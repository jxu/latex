% Pray I don't screw this test up again

\documentclass{report}  % For chapter number
\usepackage{amsmath, chngcntr}

\counterwithin{equation}{section}  % Reset equation counter every section
\counterwithout{equation}{chapter} 
\setlength\parindent{0pt}  % No indent

\begin{document}
\setcounter{chapter}{15}
\section{Double Integrals Over Rectangles}
\subsection*{Iterated Integrals}
\paragraph{Fubini's Theorem} 
allows us to switch the order of integration. For $a \le x \le b, c \le y \le d$,
\begin{equation}
	\iint \limits_R f(x,y) \, dA = 
	\int_a^b \int_c^d f(x,y) \, dy \, dx =
	\int_c^d \int_c^d f(x,y) \, dx \, dy
\end{equation}

\section{Double Integrals Over General Regions}
\subsection*{Integrals Between Curves}
\paragraph{Type I} Region lies between two functions of $x$, that is 
$$a \le x \le b, \qquad g_1(x) \le y \le g_2(x)$$

\paragraph{Type II} Region lies between two functions of $y$, that is 
$$c \le y \le d, \qquad h_1(x) \le x \le h_2(x)$$

To solve these, make sure the function bounds are in the inner integral. 
See textbook for images.

Area is defined as $\iint 1 \, dA$.

\subsection*{Switching Order of Integration}
Integrals can be switched as long as the region is the same. 
For example, the region defined by $$0 \le x \le 4, \sqrt{x} \le y \le 2$$ is the same region as $$0 \le y \le 2, 0 \le x \le y^2$$

Draw a picture!

\section{Double Integrals in Polar Coordinates}
\begin{equation}
	r^2 = x^2 + y^2 \qquad x = r \cos{\theta} \qquad y = r \cos{\theta}
\end{equation}

\begin{equation}
	\iint \limits_R f(x,y) \, dA = 
	\int_\alpha^\beta \int_a^b f(r \cos{\theta}, r \cos{\theta}) r \, dr \, d\theta
\end{equation}
The "infinitesimal rectangle" $dA = dx \, dy = r \, dr \, d\theta$.

\section{Applications of Double Integrals}
Omitted. 

\section{Surface Area}
\begin{equation}
	A = \iint \limits_D \sqrt{f_x(x,y)^2 + f_y(x,y)^2 + 1} \, dA 
\end{equation}
Note the similarity to the arc length formula.

\section{Triple Integrals}
\subsection*{Iterated Integrals}
\paragraph{Fubini's Theorem} Allows us to switch the order of integration. 
There are different types of regions as well, defined between two functions. 
Make sure the functions are in the inner integrals.


\end{document}