% Pray I don't screw this test up again

\documentclass{report}  % For chapter number
\usepackage{amsmath, chngcntr}

\counterwithin{equation}{section}  % Reset equation counter every section
\counterwithout{equation}{chapter} 
\setlength\parindent{0pt}  % No indent

\begin{document}
\setcounter{chapter}{15}
\section{Double Integrals Over Rectangles}
\subsection*{Iterated Integrals}
\paragraph{Fubini's Theorem} 
allows us to switch the order of integration. For $a \le x \le b, c \le y \le d$,
\begin{equation}
	\iint \limits_R f(x,y) \, dA = 
	\int_a^b \int_c^d f(x,y) \, dy \, dx =
	\int_c^d \int_c^d f(x,y) \, dx \, dy
\end{equation}

\section{Double Integrals Over General Regions}
\subsection*{Integrals Between Curves}
\paragraph{Type I} Region lies between two functions of $x$, that is 
$$a \le x \le b, \qquad g_1(x) \le y \le g_2(x)$$

\paragraph{Type II} Region lies between two functions of $y$, that is 
$$c \le y \le d, \qquad h_1(x) \le x \le h_2(x)$$

To solve these, make sure the function bounds are in the inner integral. 
See textbook for images.

Area is defined as $\iint 1 \, dA$.

\subsection*{Switching Order of Integration}
Integrals can be switched as long as the region is the same. 
For example, the region defined by $$0 \le x \le 4, \sqrt{x} \le y \le 2$$ is the same region as $$0 \le y \le 2, 0 \le x \le y^2$$

Draw a picture!

\section{Double Integrals in Polar Coordinates}
\begin{equation}
	r^2 = x^2 + y^2 \qquad x = r \cos{\theta} \qquad y = r \cos{\theta}
\end{equation}

\begin{equation}
	\iint \limits_R f(x,y) \, dA = 
	\int_\alpha^\beta \int_a^b f(r \cos{\theta}, r \cos{\theta}) r \, dr \, d\theta
\end{equation}
The "infinitesimal rectangle" $dA = dx \, dy = r \, dr \, d\theta$. 
\textbf{Don't forget the r!}

\section{Applications of Double Integrals}
Omitted. 

\section{Surface Area}
\begin{equation}
	A = \iint \limits_D \sqrt{f_x(x,y)^2 + f_y(x,y)^2 + 1} \, dA 
\end{equation}
Note the similarity to the arc length formula.

\section{Triple Integrals}
\subsection*{Iterated Integrals}
\paragraph{Fubini's Theorem} Allows us to switch the order of integration. 
There are different types of regions as well, defined between two functions. 
Make sure the functions are in the inner integrals.

\section{Triple Integrals in Cylindrical Coordinates}
\paragraph{Cylindrical Coordinates} Points are given as $(r, \theta, z)$. 
$r$ and $\theta$ are the polar coordinates and $z$ is the height. 

\begin{equation}
	x = r \cos{\theta} \qquad y = r\sin{\theta} \qquad z = z
\end{equation}
\begin{equation}
	r^2 = x^2 + y^2 \qquad \tan{\theta} = \frac{y}{x}
\end{equation}
\begin{equation}
	\iiint \limits_E f(x, y, z) \, dV = \int_{\theta_1}^{\theta_2} \! \int_{f_1}^{f_2} \! \int_{z_1}^{z_2}
	f(r \cos{\theta}, r\sin{\theta}, z) \, r \, dz \, dr \, d\theta
\end{equation}

\section{Triple Integrals in Spherical Coordinates}
\paragraph{Spherical Coordinates} Points are given as $(\rho, \theta, \phi)$. 
$\rho$ is the distance from the point to the origin, 
$\theta$ is the same angle as in polar coordinates (xy-plane), and
$\phi$ is the angle between the positive z-axis and the line segment to the point. \\ \\
Note $$\rho \ge 0 \qquad \text{and} \qquad 0 \le \phi \le \pi$$

A sphere is given as $\rho = r$ and a cone is given as $\phi = c$. 

\subsection*{Conversion}
\begin{equation}
	x = \rho \, \sin{\phi} \, \cos{\theta} \qquad
	y = \rho \, \sin{\phi} \, \sin{\theta} \qquad
	z = \rho \, \cos{\phi} 
\end{equation}
Remember since $\theta$ corresponds to polar coordinates, so $x$ and $y$ are just multiplied by $\rho \, \sin{\phi}$. 
See diagram in textbook.

\begin{equation}
	\rho^2 = x^2 + y^2 + z^2
\end{equation}
For $f(x, y, z) = g(\rho, \theta, \phi)$,
\begin{equation}
	\iiint \limits_E f(x, y, z) \, dV = 
	\int_{\phi_1}^{\phi_2} \! \int_{\theta_1}^{\theta_2} \! \int_{\rho_1}^{\rho_2} 
	g(\rho, \theta, \phi) \, \rho^2 \sin{\phi} \, d\rho \, d\theta \, d\phi
\end{equation}
Where $dV = \rho^2 \sin{\phi} \, d\rho \, d\theta \, d\phi$.

\section{Change of Variables in Multiple Integrals}
\subsection*{Boundaries}
To change coordinates, we consider the transformation $T$ from the uv-plane region $S$ to the xy-plane $R$:
$$T(u,v) = (x,y)$$
The inverse transformation $T^{-1}$ converts $(x, y) \to (u, v)$. 

\subsection*{The Jacobian}
To actually calculate the integral, we have to convert $dA = \mathbf{J}(u,v) \, du \, dv$,
where the \textbf{Jacobian} is defined as

\begin{equation}
	\mathbf{J}(u,v) = 
	\frac{\partial (x,y)}{\partial (u,v)} = 
	\begin{vmatrix}
		\dfrac{\partial x}{\partial u}& \dfrac{\partial x}{\partial v}\\[1em]
		\dfrac{\partial y}{\partial u}& \dfrac{\partial y}{\partial v}\\
	\end{vmatrix}
	= \frac{\partial x}{\partial u} \frac{\partial y}{\partial v} - 
	  \frac{\partial x}{\partial v} \frac{\partial y}{\partial u}
\end{equation}

To integrate a region $S$ in the uv-plane from a region $R$ in the xy-plane, 
\begin{equation}
	\iint \limits_R f(x,y) \, dA = \iint \limits_S f(x(u,v), y(u,v)) \,
	\mathbf{J}(u,v) \, du \, dv
\end{equation}

\end{document}